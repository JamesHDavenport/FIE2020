\documentclass[conference]{IEEEtran}
\IEEEoverridecommandlockouts
% The preceding line is only needed to identify funding in the first footnote. If that is unneeded, please comment it out.
\usepackage{cite}
\usepackage{amsmath,amssymb,amsfonts}
\usepackage{algorithmic}
\usepackage{graphicx}
\usepackage{textcomp}
\usepackage{xcolor}
\def\BibTeX{{\rm B\kern-.05em{\sc i\kern-.025em b}\kern-.08em
    T\kern-.1667em\lower.7ex\hbox{E}\kern-.125emX}}
\begin{document}

\title{Overcoming the Challenges of Teaching Cybersecurity in UK Computer Science Degree Programmes}

% BLIND Paper should be blinded
%\author{\IEEEauthorblockN{Tom Crick\IEEEauthorrefmark{1}, James
%    H. Davenport\IEEEauthorrefmark{2}, Paul Hanna\IEEEauthorrefmark{3}, Alastair
%    Irons\IEEEauthorrefmark{4} and Tom Prickett\IEEEauthorrefmark{5}} 
%\IEEEauthorblockA{\IEEEauthorrefmark{1}%School of Education, 
%Swansea University, Swansea, UK; Email: thomas.crick@swansea.ac.uk} 
%\IEEEauthorblockA{\IEEEauthorrefmark{2}%Dept.~Computer Science, 
%University of Bath, Bath, UK; Email: j.h.davenport@bath.ac.uk}
%\IEEEauthorblockA{\IEEEauthorrefmark{2}, 
%Ulster University, Belfast, UK; Email: jrp.hanna@ulster.ac.uk}
%\IEEEauthorblockA{\IEEEauthorrefmark{4}%Faculty of Technology, 
%Sunderland University, Sunderland, UK; Email: alastair.irons@sunderland.ac.uk}
%\IEEEauthorblockA{\IEEEauthorrefmark{5}%Dept.~Computer\&Information Sciences, 
%Northumbria University, Newcastle upon Tyne, UK; Email: tom.prickett@northumbria.ac.uk}}
\maketitle

\begin{abstract}
This Innovative Practice Full Paper explores the diversity of challenges relating to the teaching of
cybersecurity in UK higher education degree programmes, through the lens of national policy, to the
impact on pedagogy and practice. An article published in the Harvard Business Review in August 2019
argued that \textit{"Every Computer Science Degree Should Require a Course in Cybersecurity "}; in the UK, universities -- alongside government, industry and professional engineering bodies -- have been
championing this over recent years, focusing on computer science and cognate undergraduate degrees
programmes; one such professional body (BCS, The Chartered Institute for IT) has been mandating this
in accredited undergraduate degree programmes since 2015. Delivering cybersecurity effectively across
general computer science programmes presents a number of challenges related to pedagogy,
resources, faculty and infrastructure, as well as responding to industry requirements.

There is a serious demand for cybersecurity specialists, both in the UK and globally (estimates vary, but are always large – and increasing); there is thus significant and growing higher education provision
related to specialist undergraduate and postgraduate courses focusing on varying aspects of
cybersecurity (for example, cryptography, computer security, networks, digital forensics, ethical hacking, etc). To make our digital systems and products more secure, all in IT need to know some cybersecurity — thus, there is a case for depth as well as breadth; this is not a new concern, but it is a growing one. Computer science and cognate engineering disciplines are evolving to meet these demands -- both at school-level, as well as at university -- however, doing so is not without challenges. This paper explores the progress made to date in the UK, building on previous work in cybersecurity education and accreditation by highlighting key challenges and opportunities, as well as identifying a number of
enhancement activities for use by the international cybersecurity education community. It frames these
challenges through concerns with the quality and availability of underpinning educational resources, the
competencies and skills of faculty (especially focusing on pedagogy, progression and assessment), and
articulating the necessary technical resources and infrastructure related to delivering rigorous
cybersecurity content in general computer science and cognate degrees.

Though this critical evaluation of an emerging national case study of cybersecurity education in the UK,
we also present a number of recommendations across policy and practice -- from pedagogic principles
and developing effective cybersecurity teaching practice, challenges in the recruitment, retention and
professional development of faculty, to supporting diverse routes into post-compulsory cybersecurity
education (and thus, diverse careers) -- to provide the foundation for potential replicability and portability to other jurisdictions contemplating related education and skills reform initiatives and interventions.
\end{abstract}

\begin{IEEEkeywords}
cybersecurity, computer science education, curricula, accreditation, UK
\end{IEEEkeywords}

\section{Introduction}\label{sec:intro}
An article published in the Harvard Business Review in August 2019 argued that ``{\emph{Every Computer Science Degree Should Require a Course in Cybersecurity}}''\cite{cable_2019}; in the UK, universities
--- alongside government, industry and professional bodies -- have been championing this over recent years, focusing on computer science and cognate undergraduate degrees programmes. One professional body --- BCS, The Chartered Institute for IT --- has been mandating this in accredited undergraduate degree programmes since 2015\cite{Cricketal2019}. Delivering cybersecurity effectively across general computer science programmes presents a number of challenges related to pedagogy, underpinning educational resources, available skills and technical resources. This paper explores the progress to date, as well as a starting call to arms to the UK higher education sector by highlighting a number of future challenges and opportunities.

%\section{Background}
\section{Pedagogic Principles}
%To do - what are the key points?
%technical and not technical skills?%

Though it is generally thought of as part of computing, cybersecurity is actually a multidisciplinary ``subject'', or, in the words of \cite{Parrishetal2018a} a \emph{meta-discipline}. This point has been made several times, e.g.  in \cite{Stockman2013}, who described integrating Criminal Justice and Political Science into the study of cybersecurity.  
\subsection{Academic skills}\label{sec:academic}
If we look at the academic skill sets that a good Chief Information Security Officer should have, we can break them into three rough groupings.
\begin{itemize}
\item{\bf Psychology:} While the authors would not necessarily go as far as PurpleSec, who claim \cite{PurpleSec2020a} that ``98\% of cyber attacks rely on social engineering'', it is quite clear that a very large proportion do: not least the attacks classified as ``phishing'', ``spear-phishing'' and ``whaling''. The technical skills required to go phishing are minimal: being able to write
\begin{verbatim}
<a href="bad url">good url</a>
\end{verbatim}
generally suffices, and even that can be bought in.  Forging e-mail addresses is generally needed as well if one is to go spear-phishing. But the real skill comes in knowing what will get under people's radar.
\par
To defend against phishing, to inculcate good password habits\footnote{Whatever those might be: opinions vary and well-known pundits (e.g. \cite{Grimes2019b}) will disagree with NIST's advice \cite{NIST2019e}.} and much more depends on understanding, or at least following the advice of those who understand, the psychology of the user \cite{InglesantSasse2010a}. In particular it is imprtant not to fall into the ``users are the enemy'' trap \cite{AdamsSasse1999}.
\item{\bf Managerial:} Clearly the CISO has to manage the team. But there is much more than that. The CISO has to be a team player within top management. The CISO is responsible for the Cybersecurity Incident Response Plan. But \cite{Secureworks2019a} lists among its major flaws in CIRPs that they are lacking organisational support and buy-in:
\begin{itemize}
\item Plan sponsor lacks appropriate authority
(e.g., Executive Leadership Team, CIO, CTO, CISO);
\item Incident stakeholders do not know the plan exists;
\item Was developed unilaterally by a single business unit;
\item Roles and responsibilities for non-technical teams are vague.
\end{itemize}
All of these are managerial failings.
\item{\bf Technical:} there are of course many technical things that a CISO needs to be on top of.  They feature in the list \cite{Beaver2017a} of ``10 Essential Elements for Success as an Information Security Professional'' as ``(6) Find your specialty; (7) Maintain your technical edge; (8) Constantly improve your methodologies''.

These technical skills tend to be the ones that a Computer Science Department is best at teaching, though even here there are challenges --- see section \ref{practice}. One specific question is ``how much cryptography need a security expert know''. The classic answer is ``enough not to be dangerously ignorant, and not enough to be dangerously knowledgeable'', which is true but not helpful: see \S\ref{sec:second}.
\end{itemize}
\subsection{Human skills}\label{sec:human}
Besides the subject-specific skills mentioned above, there are also the human, or `soft', skills. It could be argued (e.g. \cite{Palkar2013a}) that these are underrated throughout computing education, but they are certainly necessary in cybersecurity.
\cite{Froehlich2019a} stresses them for the CISO, but the same is true throughout the cybersecurity industry. See also \cite{InfoSec2019a} --- obviously they have an axe to grind, but the message resonates with much the authors have heard, an agrees with the Wall Street Journals Cybersecurity Executive Forum \cite{WallStreetJournal2018c}. Their list of ``top five skills'' is this.
\begin{enumerate}
\item    Problem-solving;
\item    Communication;
\item    Analytical thinking;
\item    Collaboration/teamwork;
\item    Attention to detail.
\end{enumerate}

A Computer Science Department  would probably like to think that it taught most of these. Certainly a BCS-accredited one has to teach collaboration/teamwork, as that is a long-standing requirement for accreditation, despite the students' dislike of it \cite{Cricketal2020a}. However, there are doubts about the depth to which these are done, compared with industry's demands. Notably the collaboration required in cybersecurity is generally part of a multi-function team, rather than the group software engineering activity that tends to be the response to the demand to teach group working. Similarly the problem-solving required is that of being faced with an underspecified problem: ``it looks like we've been hacked'', but with a vast amount of information, most of it irrelevant.
%What else?

It is generally hard to make students take the development of these skills seriously, especially as they are hard to assess in the rigorous way computer scientists (staff and students) are used to. The authors' approaches to this is described in section \ref{sec:authors}.
\section{Developing effective cybersecurity teaching practice}\label{practice}

What is the most appropriate way to teach cybersecurity? \cite{Weiss:2013:THC:2527148.2527180} highlights there are benefits from teaching this in a practical manner. Real world case studies can be employed \cite[e.g.]{BritishAirways2018a,Zoom2020a}. Use can be made of guest lectures by industrialists to share practical insights and hence providing students with micro-exposure to the world of work is another positive contribution. One further approach is the inclusion of appropriate cybersecurity standards within the curricula.

The PCI DSS \cite{PCI2018b} is one such standard that has been used in precisely this manner. PCI DSS underpins all processing of credit/debit cards. Nevertheless, it is very rarely mentioned in generalist computer scientist courses. This would not matter so much if everyone handling payments data were sent by their employers on an effective PCI DSS course. However, the payments business is now so spread across websites, often run by small and medium enterprises (SME), or non-specialists. Even larger enterprises are not immune: \cite{BritishAirways2018a} reports that the recent British Airways breach was caused by a failure to adhere to PCI DSS in website maintenance. Section \ref{sec:second} describes one way to bring PCI DSS to life in an assignment.

Another way of adopting a more practical pedagogy is by teaching cybersecurity through the lens of hacking or the hacker curriculum \cite{bratus2010teaching}. Such an approach facilitates students to be more experimental and creative in their exploration of the discipline and can have corresponding benefits for their engagement. 


\section{challenges in the recruitment, retention and professional development of faculty}\label{sec:staff}
It is well known that cybersecurity skills are in short supply, in both industry~\cite{Ackerman2019a} and academia~\cite{schneider2013,endicott2018searching}. The demand for cybersecurity skills in industry makes it difficult for academia to attract academics with knowledge, practical experience, research background and academic aspirations. As universities expand their cybersecurity provision it is not uncommon to find multiple jobs advertised at the same time. Recent example have included a professor of cybersecurity, two senior academic positions and two junior academic positions in one advert. There are other examples in the UK of cybersecurity lecturing jobs remaining unfilled for longer than a year; there are also examples of cybersecurity research groups moving en masse from one university to another.

% this survey is quite old now - something more recent the ESG 
For example, research into the state of IT conducted annually by Enterprise Strategy Group (ESG) has revealed that the skills gap in information security continues to widen and has doubled in the past five years; in 2014, 23\% of respondents to the survey stated that their organisation had a problematic shortage of information security skills -- this had climbed to 51\% at the beginning of 2018~\cite{ESG:2018}. Clearly, cybersecurity is an issue which is being felt across many industries and organisations, and is a concern which extends beyond IT leadership into the boardroom~\cite{Ackerman2019a}.

The ESG survey is international, but ESG have confirmed that the UK figures are very similar. In the UK, there has been a resurgence of job adverts to recruit academic staff with specialisms in cybersecurity over the past three years.

%TO DO - ?CPD?  See conclusions

\section{Quality of resources to support cybersecurity education?}
Effective teaching requires appropriate supporting resources. The extent to which appropriate resources are available and suitable will be evaluated next. This evaluation highlights a number of occasions when underpinning resources could be improved. 
\subsection{Underpinning resources}
The formal resources for Cybersecurity education are in reasonable shape given the inevitable fast-moving nature of the subject. The authors see \cite{Anderson2008} as a major work, though probably daunting as a extbook. It is dated, but a new edition is in preparation, and largely on the author's website. More recently a UK-funded project has produced the ``Cyber Security Body Of Knowledge'' \cite{Bristol2019a}. This is not intended as a textbook, but is a useful reference.

One gap in the educational resources is the absence, as far as the authors know, of a good answer to ``how much cryptography need a security expert know''.

More worrying is the problem described in \cite[\S IV.B]{Cricketal2019} --- the state of general computer science educational resources with respect to cybersecurity. In particular \cite{Drop2019} describes the poor state of database textbooks withrespect to SQL injection.  Despite the fact that SQL injection is theoretically well-understood, it is still a real problem today, tweny years after it was first described: 
\begin{quote}
Overall,
SQL Injection (SQLi) accounted for more than 72%
of all attacks when looking at all verticals during this
period [December 2017--November 2019]. \cite{Akamai2020a}
\end{quote}
\subsection{Provision of laboratories}
Delivering a practical take upon cybersecurity often requires specialist computing resources, certainly if any form of penetration testing/ethical hacking is to be taught. The traditional solution to this was a dedicated laboratory, generally not connected to the outside world (in the case of the UK, the JANET network)  in order not to breach the operating conditions of the network. In practice it will probably not even be directly connected to the university's internal network for the same reasons.  An upmarket version of such a laboratory is described in \cite{Abler2006}, though an adequate one can be build for roughly \pounds10,000 in capital costs.  The real problem, which many computer science departments in the UK will struggle with, is the technician (We use the term loosely. Quite often the labour gets dumped, inappropriately, on research students. This also causes problems with maintaining the knowledge base.) time to maintain it: lost passwords, trashed machines if the hacking escapes, etc., and the problems of keeping the underlying infrastructure up-to-date with security patches while not changing the target machines the students are practising against.

An interesting alternative is to host such a laboratory ``in the cloud'', as recommended by, for example, \cite{Salah2014a}. This would eliminate the capital expenditure (in favour of recurrent cloud costs, but these should be significantly less). The impact on technician/support time is less clear. Ideally it ought to be less, but a lot depends on the extent to which the cloud provider's authentication structure can be interfaced with the host university's: the second author is currently having problems here with an unrelated piece of teaching outsourcing.

Outsourcing to the cloud means that the students' ``hacking'' commands traverse the university's and the external networks, even though then are not ``commands'' until they reach the laboratory in the cloud. Different universities in the UK, even reading the same external network's (JANET's) policies, have different views on whether this is permissible.

%Commonly such a laboratory will be not directly connected to the JANET network in order not to breach the operating conditions of the network. This creates further challenges in the form of acquisition and maintenance of specialist laboratory provision. TO DO - could benefit from a reference.  JHD: how is the above?


%\section{Research Questions}

%\section{Research Approach}

%\section{Body - Structure here to do}

\section{Teaching Innovations}\label{sec:authors}
As introduced in section \ref{sec:intro}, teaching cybersecurity well for the workplace is more than about academic skills. However, it is very hard to motivate computer scientists to study pure human skills. This is far from being a feature of students, as \cite{Beaver2019c} observes.
\begin{quote}
Still, technical knowledge seems to trump everything. I've attended national and regional information security conferences that have sessions on security careers and the essential soft skills for cybersecurity success. They're not nearly as well attended as the sessions on cool and sexy topics like threat hunting, cryptocurrency and ransomware.
\end{quote}
Hence the ideal assignment in cybersecurity mixes the academic and human skills, preferably inseparably.
\subsection{Second Author}\label{sec:second} %JHD
This author teaches on a Cybersecurity module in a general computing degree. As such there is a wide range of backgrounds on the course. We asked in section \ref{sec:academic} ``how much cryptography need a security expert know''. The answer for this course is ``about 90 minutes worth'', which seems to cover the bases well enough: in particular enough on cryptographic hashing to imderpin the lecture on password management.


Since Cybersecurity is a practical subject, much of the learning takes place through the coursework. The assessment for the module is structured with these weightings.
\begin{itemize}
\item[20]Class test --- basically a traditional examination. Normally a closed-book test, but this year, due to Covid-19, it will be sat by students who have dispersed, so will have to be open-book.
\item[30]Group presentation. The class divide themselves into groups of 4--5, and each group picks a topic from either the ``OWASP Top 10'' \cite{OWASP2017a} or the ``OWASP Mobile Top  10'' \cite{OWASP2016a} and does a presentation on it to the rest of the class.  In terms of the human skills identified in section \ref{sec:human}, this is meant to help with the communication and team-working skills. Indeed, Covid-19 and the consequent dispersal of the students has meant that they are also learning remote team working: an unexpected bonus.  The author has found that it is extremely important to share this rationale, and the evidence at \cite{WallStreetJournal2018c,InfoSec2019a,Beaver2019c} with them.  Industrial speakers are also asked to stress these points.
\item[50]The students have to analyse three different (i.e. different vendors and style) on-line purchases (or mock-purchases, they are allowed to use an invalid  credit card) made subject to PCI DSS, looking at the screen as the ordinary customer would see it, but also at the browser logs (HAR files) and network logs (Wireshark or equivalent). For each, they are asked the following.
\begin{enumerate}
\item With which websites does your browser communicate during the transaction? Are there any that worry you, or whose function you do not understand?
\item[*]Last year two different students, purchasing from two different UK-based sites, found \verb+yandex.ru+ appearing here.
\item Looking at the logs, to which sites does your payment card number get sent, and how is it protected in transit? You \emph{should} quote the relevant part of the logs, but \emph{should} replace the card number and any other identifying/sensitive data, e.g. by \verb+NNNN NNNN NNNN NNNN+, before quoting.
\item Looking at the HTML you have saved, do you feel confident you know what it is doing with your data?
\item How dependent is the HTML you have on the correct functioning of the DNS? In particular, could bad DNS results result in a security problem?\label{DNS}
\item What makes you think that the sum of money displayed to you is the sum that will be transmitted to your bank?
\end{enumerate}
Question \ref{DNS} is meant to help them realise the interconnected nature of today's Internet, and the hidden dependencies. They have had a lecture about the DNS, and in particular the recent wave of home router attacks  \cite[and others]{Balaji2020a}.
\par
In addition, they are asked these overall questions.
\begin{enumerate}
\item What have you learned about the security of your card data?
\item In particular, what did you learn from the logs/HTML that you could not have reasonably deduced as a shopper with no access to these?
\item How obvious is the security of the websites to the shopper?
\item How might the system be more transparent to the shopper?
\end{enumerate}
Questions 3 and 4 are really open-ended ``problem-solving'' questions, to which the author doesn't have a neat solution. This really frustrates students, who want to know ``the right answer'', but is much closer to reality.
\end{itemize}
\subsection{Fourth Author}\label{sec:fourth} % AI
This author teaches on dedicated BSc and MSc courses in Cybersecurity. The author has employed a number of assessment instruments in cybersecurity to encourage students to use the assessment activities to enable student learning, described more fully in [blinded]. 
%BLIND \cite{Irons2019a}
The use of scenarios to encourage students to think both as attackers and defenders has been particularly helpful in enabling students to understand the cybersecurity environment. As an overall assessment strategy in cybersecurity attempts should be made to: 
\begin{itemize}
\item get students to identify and critically evaluate threats – ranging from nuisance threats to “advanced persistent threats”;
\item design, develop and implement strategies to counter the threats;
\item identify when breaches or attacks have taken place and critically evaluate the impact of those;
\item design, develop and implement approaches to recover from attack;
\item give students the opportunity to evaluate attacks and develop more robust cybersecurity defences as a result.
\end{itemize}

The above can be done in the context of specific cybersecurity scenarios or case studies, but can also be utilised to encourage students to think about and present policies and procedures for cybersecurity environments. 

One particular assessment task that has been used by the author which helps pull together many of the cybersecurity threads and complexities is the use of ``infographs''. An extract from an assessment using infographs is given below.
\begin{quote}
The cybersecurity environment is a wide and complex one. For the first part of this assignment you are required to produce an infograph (1 page) outlining the typical threats that either a) individuals in society or b) organisations face from breaches of cybersecurity. The design of your infograph and the content of the infograph is left to you to decide but you should consider visual impact, key messages, data to support, examples and underpinning research. You will have the chance to present your infograph to your peers, academics and guests from industry. You should be able to discuss the points raised on your infograph, explaining the detail and answering any questions asked. 
\end{quote}
As well as allowing the student to analyse and evaluate a particular issue or concern the assessment enables the development and assessment of a series of professional competencies (see section \ref{sec:human}), including communication, presentation skills, and the summarising of complex cybersecurity issues. 

\subsection{Fifth Author}\label{sec:fifth} % TP

TP to complete.

\section{Conclusions and Further Work}

In the previous section, we see how some of the authors try to produce assignments which exercise both the academic and human skills needs. 

On the practical side, the community could do much to help itself in the way of sharing best practice.
\begin{enumerate}
\item Given the shortage of staff in the area (section \ref{sec:staff}), and the fact that universities have to teach this with less-than-ideally qualified staff, there is a real need for coordinated professional development in this area.
\item Sharing best  practice with respect to physical laboratories, especially reducing technical support effort.
\item The same for in-cloud laboratories.
\item At least in the UK, getting university lawyers to form a consistent view on the legitimacy of outsourcing cybersecurity laboratories to the cloud
\end{enumerate}


\section*{Acknowledgement}

%The authors wish to thank Sally Pearce, Academic Accreditation Manager at BCS, The Chartered Institute for IT for supplying the summary information related to accreditation of UK degree programmes. Many people, accreditors and accredited, have contributed to accreditation practice in the UK (and elsewhere), and spreading good practice.

Removed for blind refereeing. %BLIND
%The first, second, fourth and fifth authors' institutions are members of the Institute of Coding, an initiative funded by the Office for Students (England) and the Higher Education Funding Council for Wales.


%\section*{References}

\bibliographystyle{IEEEtran}
%\bibliography{security}
\bibliography{IEEEabrv,security}





\end{document}
